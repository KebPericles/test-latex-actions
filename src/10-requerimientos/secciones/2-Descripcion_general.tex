\section{Descripción general}
\subsection{Perspectiva del producto}
% TODO

\subsection{Funciones del producto}
% TODO Funciones del producto a grandes rasgos
Funcionalidades Incluidas:

\begin{itemize}
        \item Registro y autenticación de usuarios.
        \item Visualización de horarios personales para usuarios registrados.
        \item Gestión de información sobre edificios, incluyendo nombre, ubicación y asociación con salones y laboratorios.

        \item Gestión de información sobre salones y laboratorios, incluyendo número, capacidad, tipo y disponibilidad en función de horarios preestablecidos.

        \item Gestión de información detallada sobre materias, incluyendo nombre de la materia, profesor asignado, grupos y horarios flexibles.

        \item Cálculo automático de horarios de clases basado en la información de materias ingresada por usuarios registrados.
\end{itemize}


Funcionalidades Excluidas:

La importación automática de horarios desde el sistema SAES de la escuela (esto se considerará como una mejora futura).


\subsection{Características de los usuarios}
% TODO
\begin{itemize}
        \item Usuarios Comunes: Estudiantes de la universidad que deseen registrarse y utilizar la plataforma para acceder a información sobre sus horarios de clases.
        \item Administradores: Personal autorizado de la universidad que gestionará la información sobre edificios, salones, laboratorios y materias.
\end{itemize}

\subsection{Restricciones}
% TODO Limitaciones impuestas en los desarrolladores

\subsection{Suposiciones y dependencias}
% TODO Suposiciones que si cambian pueden afectar a los requisitos
Se asume que los datos proporcionados por los administradores y usuarios son precisos y actualizados.

\subsection{Requisitos futuros}
% TODO
La integración con el sistema SAES está sujeta a futuras consideraciones y recursos disponibles.
Amet ullamco ex ea non laboris id esse dolore quis commodo nostrud mollit fugiat tempor.